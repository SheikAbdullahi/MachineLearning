\documentclass{article}
\usepackage{hyperref} % Needed to create hyperlinks
\usepackage{graphicx} % Needed to insert images

\begin{document}

\title{Resolution of Code Issues in Machine Learning Project}
\author{}
\date{}
\maketitle

\section*{Part a}

In regression analysis, scaling is of paramount importance as it mitigates the influence of the magnitude of different features on the model. This process is crucial for models sensitive to the scale of the input features. In this implementation, the \textit{Standard Scaler} is utilized, which scales the features by subtracting the mean and scaling to unit variance, ensuring all features contribute equally to the model's performance.

Furthermore, the approach to data splitting is pivotal in model training and evaluation. In this instance, $80\%$ of the data is allocated as training data, and the remaining $20\%$ is used as test data. This ratio is a widely adopted convention in machine learning, enabling the model to learn effectively from the majority of the data and subsequently be evaluated on unseen data to assess its generalization performance. Such a split aims to ensure a balance between maximizing the model's learning and retaining sufficient unseen data for a reliable performance evaluation.

The code is made available on GitHub and can be accessed through the following link: \href{https://github.com/SheikAbdullahi/MachineLearning}{GitHub Repository}.

\begin{figure}[htbp]
    \centering
    \includegraphics[width=0.55\textwidth]{fig/Project1Problem1Fig1ML.png}
    \caption{Visual representation}
    \label{fig:python-code}
\end{figure}

\end{document}
